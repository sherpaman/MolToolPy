
% Default to the notebook output style

    


% Inherit from the specified cell style.




    
\documentclass{report}

    
    
    \usepackage{graphicx} % Used to insert images
    \usepackage{adjustbox} % Used to constrain images to a maximum size 
    \usepackage{color} % Allow colors to be defined
    \usepackage{enumerate} % Needed for markdown enumerations to work
    \usepackage{geometry} % Used to adjust the document margins
    \usepackage{amsmath} % Equations
    \usepackage{amssymb} % Equations
    \usepackage{eurosym} % defines \euro
    \usepackage[mathletters]{ucs} % Extended unicode (utf-8) support
    \usepackage[utf8x]{inputenc} % Allow utf-8 characters in the tex document
    \usepackage{fancyvrb} % verbatim replacement that allows latex
    \usepackage{grffile} % extends the file name processing of package graphics 
                         % to support a larger range 
    % The hyperref package gives us a pdf with properly built
    % internal navigation ('pdf bookmarks' for the table of contents,
    % internal cross-reference links, web links for URLs, etc.)
    \usepackage{hyperref}
    \usepackage{longtable} % longtable support required by pandoc >1.10
    \usepackage{booktabs}  % table support for pandoc > 1.12.2
    

    
    
    \definecolor{orange}{cmyk}{0,0.4,0.8,0.2}
    \definecolor{darkorange}{rgb}{.71,0.21,0.01}
    \definecolor{darkgreen}{rgb}{.12,.54,.11}
    \definecolor{myteal}{rgb}{.26, .44, .56}
    \definecolor{gray}{gray}{0.45}
    \definecolor{lightgray}{gray}{.95}
    \definecolor{mediumgray}{gray}{.8}
    \definecolor{inputbackground}{rgb}{.95, .95, .85}
    \definecolor{outputbackground}{rgb}{.95, .95, .95}
    \definecolor{traceback}{rgb}{1, .95, .95}
    % ansi colors
    \definecolor{red}{rgb}{.6,0,0}
    \definecolor{green}{rgb}{0,.65,0}
    \definecolor{brown}{rgb}{0.6,0.6,0}
    \definecolor{blue}{rgb}{0,.145,.698}
    \definecolor{purple}{rgb}{.698,.145,.698}
    \definecolor{cyan}{rgb}{0,.698,.698}
    \definecolor{lightgray}{gray}{0.5}
    
    % bright ansi colors
    \definecolor{darkgray}{gray}{0.25}
    \definecolor{lightred}{rgb}{1.0,0.39,0.28}
    \definecolor{lightgreen}{rgb}{0.48,0.99,0.0}
    \definecolor{lightblue}{rgb}{0.53,0.81,0.92}
    \definecolor{lightpurple}{rgb}{0.87,0.63,0.87}
    \definecolor{lightcyan}{rgb}{0.5,1.0,0.83}
    
    % commands and environments needed by pandoc snippets
    % extracted from the output of `pandoc -s`
    \providecommand{\tightlist}{%
      \setlength{\itemsep}{0pt}\setlength{\parskip}{0pt}}
    \DefineVerbatimEnvironment{Highlighting}{Verbatim}{commandchars=\\\{\}}
    % Add ',fontsize=\small' for more characters per line
    \newenvironment{Shaded}{}{}
    \newcommand{\KeywordTok}[1]{\textcolor[rgb]{0.00,0.44,0.13}{\textbf{{#1}}}}
    \newcommand{\DataTypeTok}[1]{\textcolor[rgb]{0.56,0.13,0.00}{{#1}}}
    \newcommand{\DecValTok}[1]{\textcolor[rgb]{0.25,0.63,0.44}{{#1}}}
    \newcommand{\BaseNTok}[1]{\textcolor[rgb]{0.25,0.63,0.44}{{#1}}}
    \newcommand{\FloatTok}[1]{\textcolor[rgb]{0.25,0.63,0.44}{{#1}}}
    \newcommand{\CharTok}[1]{\textcolor[rgb]{0.25,0.44,0.63}{{#1}}}
    \newcommand{\StringTok}[1]{\textcolor[rgb]{0.25,0.44,0.63}{{#1}}}
    \newcommand{\CommentTok}[1]{\textcolor[rgb]{0.38,0.63,0.69}{\textit{{#1}}}}
    \newcommand{\OtherTok}[1]{\textcolor[rgb]{0.00,0.44,0.13}{{#1}}}
    \newcommand{\AlertTok}[1]{\textcolor[rgb]{1.00,0.00,0.00}{\textbf{{#1}}}}
    \newcommand{\FunctionTok}[1]{\textcolor[rgb]{0.02,0.16,0.49}{{#1}}}
    \newcommand{\RegionMarkerTok}[1]{{#1}}
    \newcommand{\ErrorTok}[1]{\textcolor[rgb]{1.00,0.00,0.00}{\textbf{{#1}}}}
    \newcommand{\NormalTok}[1]{{#1}}
    
    % Define a nice break command that doesn't care if a line doesn't already
    % exist.
    \def\br{\hspace*{\fill} \\* }
    % Math Jax compatability definitions
    \def\gt{>}
    \def\lt{<}
    % Document parameters
    \title{TimeSer}
    
    
    

    % Pygments definitions
    
\makeatletter
\def\PY@reset{\let\PY@it=\relax \let\PY@bf=\relax%
    \let\PY@ul=\relax \let\PY@tc=\relax%
    \let\PY@bc=\relax \let\PY@ff=\relax}
\def\PY@tok#1{\csname PY@tok@#1\endcsname}
\def\PY@toks#1+{\ifx\relax#1\empty\else%
    \PY@tok{#1}\expandafter\PY@toks\fi}
\def\PY@do#1{\PY@bc{\PY@tc{\PY@ul{%
    \PY@it{\PY@bf{\PY@ff{#1}}}}}}}
\def\PY#1#2{\PY@reset\PY@toks#1+\relax+\PY@do{#2}}

\def\PY@tok@gd{\def\PY@tc##1{\textcolor[rgb]{0.63,0.00,0.00}{##1}}}
\def\PY@tok@gu{\let\PY@bf=\textbf\def\PY@tc##1{\textcolor[rgb]{0.50,0.00,0.50}{##1}}}
\def\PY@tok@gt{\def\PY@tc##1{\textcolor[rgb]{0.00,0.25,0.82}{##1}}}
\def\PY@tok@gs{\let\PY@bf=\textbf}
\def\PY@tok@gr{\def\PY@tc##1{\textcolor[rgb]{1.00,0.00,0.00}{##1}}}
\def\PY@tok@cm{\let\PY@it=\textit\def\PY@tc##1{\textcolor[rgb]{0.25,0.50,0.50}{##1}}}
\def\PY@tok@vg{\def\PY@tc##1{\textcolor[rgb]{0.10,0.09,0.49}{##1}}}
\def\PY@tok@m{\def\PY@tc##1{\textcolor[rgb]{0.40,0.40,0.40}{##1}}}
\def\PY@tok@mh{\def\PY@tc##1{\textcolor[rgb]{0.40,0.40,0.40}{##1}}}
\def\PY@tok@go{\def\PY@tc##1{\textcolor[rgb]{0.50,0.50,0.50}{##1}}}
\def\PY@tok@ge{\let\PY@it=\textit}
\def\PY@tok@vc{\def\PY@tc##1{\textcolor[rgb]{0.10,0.09,0.49}{##1}}}
\def\PY@tok@il{\def\PY@tc##1{\textcolor[rgb]{0.40,0.40,0.40}{##1}}}
\def\PY@tok@cs{\let\PY@it=\textit\def\PY@tc##1{\textcolor[rgb]{0.25,0.50,0.50}{##1}}}
\def\PY@tok@cp{\def\PY@tc##1{\textcolor[rgb]{0.74,0.48,0.00}{##1}}}
\def\PY@tok@gi{\def\PY@tc##1{\textcolor[rgb]{0.00,0.63,0.00}{##1}}}
\def\PY@tok@gh{\let\PY@bf=\textbf\def\PY@tc##1{\textcolor[rgb]{0.00,0.00,0.50}{##1}}}
\def\PY@tok@ni{\let\PY@bf=\textbf\def\PY@tc##1{\textcolor[rgb]{0.60,0.60,0.60}{##1}}}
\def\PY@tok@nl{\def\PY@tc##1{\textcolor[rgb]{0.63,0.63,0.00}{##1}}}
\def\PY@tok@nn{\let\PY@bf=\textbf\def\PY@tc##1{\textcolor[rgb]{0.00,0.00,1.00}{##1}}}
\def\PY@tok@no{\def\PY@tc##1{\textcolor[rgb]{0.53,0.00,0.00}{##1}}}
\def\PY@tok@na{\def\PY@tc##1{\textcolor[rgb]{0.49,0.56,0.16}{##1}}}
\def\PY@tok@nb{\def\PY@tc##1{\textcolor[rgb]{0.00,0.50,0.00}{##1}}}
\def\PY@tok@nc{\let\PY@bf=\textbf\def\PY@tc##1{\textcolor[rgb]{0.00,0.00,1.00}{##1}}}
\def\PY@tok@nd{\def\PY@tc##1{\textcolor[rgb]{0.67,0.13,1.00}{##1}}}
\def\PY@tok@ne{\let\PY@bf=\textbf\def\PY@tc##1{\textcolor[rgb]{0.82,0.25,0.23}{##1}}}
\def\PY@tok@nf{\def\PY@tc##1{\textcolor[rgb]{0.00,0.00,1.00}{##1}}}
\def\PY@tok@si{\let\PY@bf=\textbf\def\PY@tc##1{\textcolor[rgb]{0.73,0.40,0.53}{##1}}}
\def\PY@tok@s2{\def\PY@tc##1{\textcolor[rgb]{0.73,0.13,0.13}{##1}}}
\def\PY@tok@vi{\def\PY@tc##1{\textcolor[rgb]{0.10,0.09,0.49}{##1}}}
\def\PY@tok@nt{\let\PY@bf=\textbf\def\PY@tc##1{\textcolor[rgb]{0.00,0.50,0.00}{##1}}}
\def\PY@tok@nv{\def\PY@tc##1{\textcolor[rgb]{0.10,0.09,0.49}{##1}}}
\def\PY@tok@s1{\def\PY@tc##1{\textcolor[rgb]{0.73,0.13,0.13}{##1}}}
\def\PY@tok@sh{\def\PY@tc##1{\textcolor[rgb]{0.73,0.13,0.13}{##1}}}
\def\PY@tok@sc{\def\PY@tc##1{\textcolor[rgb]{0.73,0.13,0.13}{##1}}}
\def\PY@tok@sx{\def\PY@tc##1{\textcolor[rgb]{0.00,0.50,0.00}{##1}}}
\def\PY@tok@bp{\def\PY@tc##1{\textcolor[rgb]{0.00,0.50,0.00}{##1}}}
\def\PY@tok@c1{\let\PY@it=\textit\def\PY@tc##1{\textcolor[rgb]{0.25,0.50,0.50}{##1}}}
\def\PY@tok@kc{\let\PY@bf=\textbf\def\PY@tc##1{\textcolor[rgb]{0.00,0.50,0.00}{##1}}}
\def\PY@tok@c{\let\PY@it=\textit\def\PY@tc##1{\textcolor[rgb]{0.25,0.50,0.50}{##1}}}
\def\PY@tok@mf{\def\PY@tc##1{\textcolor[rgb]{0.40,0.40,0.40}{##1}}}
\def\PY@tok@err{\def\PY@bc##1{\fcolorbox[rgb]{1.00,0.00,0.00}{1,1,1}{##1}}}
\def\PY@tok@kd{\let\PY@bf=\textbf\def\PY@tc##1{\textcolor[rgb]{0.00,0.50,0.00}{##1}}}
\def\PY@tok@ss{\def\PY@tc##1{\textcolor[rgb]{0.10,0.09,0.49}{##1}}}
\def\PY@tok@sr{\def\PY@tc##1{\textcolor[rgb]{0.73,0.40,0.53}{##1}}}
\def\PY@tok@mo{\def\PY@tc##1{\textcolor[rgb]{0.40,0.40,0.40}{##1}}}
\def\PY@tok@kn{\let\PY@bf=\textbf\def\PY@tc##1{\textcolor[rgb]{0.00,0.50,0.00}{##1}}}
\def\PY@tok@mi{\def\PY@tc##1{\textcolor[rgb]{0.40,0.40,0.40}{##1}}}
\def\PY@tok@gp{\let\PY@bf=\textbf\def\PY@tc##1{\textcolor[rgb]{0.00,0.00,0.50}{##1}}}
\def\PY@tok@o{\def\PY@tc##1{\textcolor[rgb]{0.40,0.40,0.40}{##1}}}
\def\PY@tok@kr{\let\PY@bf=\textbf\def\PY@tc##1{\textcolor[rgb]{0.00,0.50,0.00}{##1}}}
\def\PY@tok@s{\def\PY@tc##1{\textcolor[rgb]{0.73,0.13,0.13}{##1}}}
\def\PY@tok@kp{\def\PY@tc##1{\textcolor[rgb]{0.00,0.50,0.00}{##1}}}
\def\PY@tok@w{\def\PY@tc##1{\textcolor[rgb]{0.73,0.73,0.73}{##1}}}
\def\PY@tok@kt{\def\PY@tc##1{\textcolor[rgb]{0.69,0.00,0.25}{##1}}}
\def\PY@tok@ow{\let\PY@bf=\textbf\def\PY@tc##1{\textcolor[rgb]{0.67,0.13,1.00}{##1}}}
\def\PY@tok@sb{\def\PY@tc##1{\textcolor[rgb]{0.73,0.13,0.13}{##1}}}
\def\PY@tok@k{\let\PY@bf=\textbf\def\PY@tc##1{\textcolor[rgb]{0.00,0.50,0.00}{##1}}}
\def\PY@tok@se{\let\PY@bf=\textbf\def\PY@tc##1{\textcolor[rgb]{0.73,0.40,0.13}{##1}}}
\def\PY@tok@sd{\let\PY@it=\textit\def\PY@tc##1{\textcolor[rgb]{0.73,0.13,0.13}{##1}}}

\def\PYZbs{\char`\\}
\def\PYZus{\char`\_}
\def\PYZob{\char`\{}
\def\PYZcb{\char`\}}
\def\PYZca{\char`\^}
\def\PYZsh{\char`\#}
\def\PYZpc{\char`\%}
\def\PYZdl{\char`\$}
\def\PYZti{\char`\~}
% for compatibility with earlier versions
\def\PYZat{@}
\def\PYZlb{[}
\def\PYZrb{]}
\makeatother


    % Exact colors from NB
    \definecolor{incolor}{rgb}{0.0, 0.0, 0.5}
    \definecolor{outcolor}{rgb}{0.545, 0.0, 0.0}



    
    % Prevent overflowing lines due to hard-to-break entities
    \sloppy 
    % Setup hyperref package
    \hypersetup{
      breaklinks=true,  % so long urls are correctly broken across lines
      colorlinks=true,
      urlcolor=blue,
      linkcolor=darkorange,
      citecolor=darkgreen,
      }
    % Slightly bigger margins than the latex defaults
    
    \geometry{verbose,tmargin=1in,bmargin=1in,lmargin=1in,rmargin=1in}
    
    

    \begin{document}
    
    
    
    \maketitle
    
    
    \tableofcontents


    
\chapter{Scripts}

\section{Calculate Mutual Info}

The script \textbf{``calculate\_mutual\_info.py''} takes as an input a
file containing various time-series replicas: each column will be
interpreted as different replica and each row will be a different value
as a function of time.

The replicas needs to have the same number of time-measures (i.e.~same
number of rows).

The output will contain a symmetric matrix of size (N x N) where N =
number of replicas, which contains the Mutual Information of each
replica against the others (on the diagonal the values of Information
Entropy of each replica).

The script starts by loading the needed packages:

    \begin{Verbatim}[commandchars=\\\{\}]
{\color{incolor}In [{\color{incolor} }]:} \PY{k+kn}{import} \PY{n+nn}{ts}
        \PY{k+kn}{import} \PY{n+nn}{matplotlib.pyplot} \PY{k+kn}{as} \PY{n+nn}{plt}
        \PY{k+kn}{import} \PY{n+nn}{numpy} \PY{k+kn}{as} \PY{n+nn}{np}
        \PY{k+kn}{from} \PY{n+nn}{argparse} \PY{k+kn}{import} \PY{n}{ArgumentParser}
\end{Verbatim}

then the argument parser is defined:

    \begin{Verbatim}[commandchars=\\\{\}]
{\color{incolor}In [{\color{incolor} }]:} \PY{n}{parser} \PY{o}{=} \PY{n}{ArgumentParser}\PY{p}{(} \PY{n}{description} \PY{o}{=} \PY{l+s}{'}\PY{l+s}{Calculate Mutual Information}\PY{l+s}{'}\PY{p}{)}
        \PY{c}{\PYZsh{}}
        \PY{c}{\PYZsh{} SEE FILE FOR DETAILS.}
        \PY{c}{\PYZsh{}}
        \PY{n}{options} \PY{o}{=} \PY{n}{parser}\PY{o}{.}\PY{n}{parse\PYZus{}args}\PY{p}{(}\PY{p}{)}
\end{Verbatim}

\subsection{Arguments}

The Input file format has been already described. Other options give the
possibility to :

\begin{itemize}
\item
  load and analyse the time-series using only one every n-th frame
  (\textbf{---stride})
\item
  define the number of bins to be used to build the histograms
  (\textbf{---nbins})
\item
  use a simple (and not so clever) optimization for calculate the
  optimal bin-width (\textbf{---opt})
\item
  specify the dimensionality and the organization of the data in the
  input file (\textbf{---ndim} and \textbf{---interleave})
  \begin{itemize}
  \item
    For more informations concerning this aspect read the next paragraph
  \end{itemize}
\item
  create an image containing a representation of the results
  (\textbf{---plot})
\end{itemize}

\subsubsection{Data dimensionality and reorganization}

By default the program assumes that the data are 1-dimensional time
series, so if the input files contains N columns it will generate N
replicas. But the data can also be multi dimensional: if the user
specify that the data are k-dimensional, if the input files contains N
columns it will generate N/k replicas. In the case the user specifies
that the data has to be represented in k($>1$) dimensions, by default
the script assumes that the values of the various dimensions of a given
replicas are consecutives columns.

\textbf{EXAMPLE:} If we specify ---dim 3 and tha files contains 6
columns, the program will genrate 2 3-dim replicas, and it will assume
that the column in the input file are:

X1 Y1 Z1 X2 Y2 Z2

i.e. : the 1-st column is the 1-st dimension of the 1-st replica, the
2-nd column in the 2-nd dimension of the 1-st replica and so on.

Specifing the option \textbf{---interleave}, the user can modify this
behaviour and the script will instead assume that the input data are
organized as the following:

X1 X2 Y1 Y2 Z1 Z2

i.e. : the first N/k colum are the 1-st dimension of replicas, followed
by N/K columns containing the 2-nd dimension and son on.

\subsection{Description}

The reorganization of the data in the correct order is made using the
following function:

    \begin{Verbatim}[commandchars=\\\{\}]
{\color{incolor}In [{\color{incolor} }]:} \PY{k}{def} \PY{n+nf}{interleave}\PY{p}{(}\PY{n}{data}\PY{p}{,}\PY{n}{ndim}\PY{p}{)}\PY{p}{:}
            \PY{n}{nfr}\PY{p}{,} \PY{n}{nrep} \PY{o}{=} \PY{n}{data}\PY{o}{.}\PY{n}{shape}	
            \PY{n}{out} \PY{o}{=} \PY{n}{np}\PY{o}{.}\PY{n}{zeros}\PY{p}{(}\PY{n}{data}\PY{o}{.}\PY{n}{shape}\PY{p}{)}
            \PY{k}{for} \PY{n}{i} \PY{o+ow}{in} \PY{n+nb}{range}\PY{p}{(}\PY{n}{nrep}\PY{o}{/}\PY{n}{ndim}\PY{p}{)}\PY{p}{:}
                \PY{o+ow}{or} \PY{n}{j} \PY{o+ow}{in} \PY{n+nb}{range}\PY{p}{(}\PY{n}{ndim}\PY{p}{)}\PY{p}{:}
                \PY{n}{out}\PY{p}{[}\PY{p}{:}\PY{p}{,}\PY{n}{ndim}\PY{o}{*}\PY{n}{i}\PY{o}{+}\PY{n}{j}\PY{p}{]}   \PY{o}{=} \PY{n}{data}\PY{p}{[}\PY{p}{:}\PY{p}{,}\PY{n}{j}\PY{o}{*}\PY{p}{(}\PY{n}{nrep}\PY{o}{/}\PY{n}{ndim}\PY{p}{)}\PY{o}{+}\PY{n}{i}\PY{p}{]}
            \PY{k}{return} \PY{n}{out}
\end{Verbatim}

Following our exploration of the script we now enter in the actual
execution.

Firstly the options are stored in more readable variables.

    \begin{Verbatim}[commandchars=\\\{\}]
{\color{incolor}In [{\color{incolor} }]:} \PY{n}{f\PYZus{}dat} \PY{o}{=} \PY{n}{options}\PY{o}{.}\PY{n}{dat}
        \PY{n}{f\PYZus{}out} \PY{o}{=} \PY{n}{options}\PY{o}{.}\PY{n}{out}
        \PY{n}{stride} \PY{o}{=} \PY{n}{options}\PY{o}{.}\PY{n}{stride}
\end{Verbatim}

and finally the data is read from the file specified from the user and
if the \emph{---interleave} option has been selected the data is
reorganized as described above and stored in a numpy array named
\emph{dat}

    \begin{Verbatim}[commandchars=\\\{\}]
{\color{incolor}In [{\color{incolor} }]:} \PY{n}{dat}   \PY{o}{=} \PY{n}{np}\PY{o}{.}\PY{n}{loadtxt}\PY{p}{(}\PY{n}{f\PYZus{}dat}\PY{p}{)}
        \PY{n}{dat}   \PY{o}{=} \PY{n}{dat}\PY{p}{[}\PY{p}{:}\PY{p}{:}\PY{n}{stride}\PY{p}{]}
        
        \PY{k}{if} \PY{p}{(}\PY{n}{options}\PY{o}{.}\PY{n}{interleave}\PY{p}{)} \PY{o}{&} \PY{p}{(}\PY{n}{options}\PY{o}{.}\PY{n}{ndim} \PY{o}{!=} \PY{l+m+mi}{1}\PY{p}{)}\PY{p}{:}
                \PY{n}{dat} \PY{o}{=} \PY{n}{interleave}\PY{p}{(}\PY{n}{dat}\PY{p}{,}\PY{n}{options}\PY{o}{.}\PY{n}{ndim}\PY{p}{)}
\end{Verbatim}

Then the dat array is used to create an instance of the TimeSer Object
defined in the \emph{ts} module.

    \begin{Verbatim}[commandchars=\\\{\}]
{\color{incolor}In [{\color{incolor} }]:} \PY{n}{DATA}\PY{o}{=} \PY{n}{ts}\PY{o}{.}\PY{n}{TimeSer}\PY{p}{(}\PY{n}{dat}\PY{p}{,}\PY{n+nb}{len}\PY{p}{(}\PY{n}{dat}\PY{p}{)}\PY{p}{,}\PY{n}{dim}\PY{o}{=}\PY{n}{options}\PY{o}{.}\PY{n}{ndim}\PY{p}{,}\PY{n}{nbins}\PY{o}{=}\PY{n}{options}\PY{o}{.}\PY{n}{nbins}\PY{p}{)}
        \PY{n}{DATA}\PY{o}{.}\PY{n}{calc\PYZus{}bins}\PY{p}{(}\PY{n}{opt}\PY{o}{=}\PY{n}{options}\PY{o}{.}\PY{n}{opt}\PY{p}{)}
\end{Verbatim}

In the TimeSer Object a series of procedre for the calculation of
Statistical Entropy, Mutual Information and Information Transfer are
available.

The most crucial and time consuming functions multiple are actually
wrapper for \textbf{FORTRAN90} code that have been compiled with the
\textbf{f2py} tool. These function variants are identified by the
post-fix \textbf{``for''}.

Some of the functions have also been parallelized with \textbf{OpenMP}.
The Module automatically identify the number of processors available on
the computer, and automatically gnereates an equal number of threads and
equally distributes the calculations among these threads. The
parallelized versions are identified by the post-fix \textbf{``omp''}.

In the script here presented we use the \textbf{``OMP''} version of the
\textbf{``mutual\_info()''} function {[}called
\textbf{``mutual\_info\_omp()''}{]}, wich produces as an output two
numpy arrays:

\begin{itemize}
\item
  \textbf{M} {[} size (NxN) N = num. of replicas {]} : Mutual
  Information.
\item
  \textbf{E} {[} size (NxN) N = num. of replicas {]} : Entropies joint
  distributions of replicas.
\end{itemize}

    \begin{Verbatim}[commandchars=\\\{\}]
{\color{incolor}In [{\color{incolor} }]:} \PY{n}{M}\PY{p}{,} \PY{n}{E} \PY{o}{=} \PY{n}{DATA}\PY{o}{.}\PY{n}{mutual\PYZus{}info\PYZus{}omp}\PY{p}{(}\PY{p}{)}
\end{Verbatim}

Then finally an image representing the Mutual Information matrix is
generated using matplotlib.

    \begin{Verbatim}[commandchars=\\\{\}]
{\color{incolor}In [{\color{incolor} }]:} \PY{n}{fig} \PY{o}{=} \PY{n}{plt}\PY{o}{.}\PY{n}{figure}\PY{p}{(}\PY{p}{)}
        \PY{n}{ax}  \PY{o}{=} \PY{n}{fig}\PY{o}{.}\PY{n}{add\PYZus{}subplot}\PY{p}{(}\PY{l+m+mi}{111}\PY{p}{)}
        \PY{n}{mat} \PY{o}{=} \PY{n}{ax}\PY{o}{.}\PY{n}{matshow}\PY{p}{(}\PY{n}{M}\PY{p}{)}
        \PY{n}{fig}\PY{o}{.}\PY{n}{colorbar}\PY{p}{(}\PY{n}{mat}\PY{p}{)}
        \PY{n}{plt}\PY{o}{.}\PY{n}{show}\PY{p}{(}\PY{p}{)}
\end{Verbatim}

If asked the image is also saved to a file in the SVG format (Scalable
Vector Graphics) that can be easily opened with any vector graphic
editor (e.g.~Inkscape, Adobe Illustrator)

    \begin{Verbatim}[commandchars=\\\{\}]
{\color{incolor}In [{\color{incolor} }]:} \PY{k}{if} \PY{n}{options}\PY{o}{.}\PY{n}{plot}\PY{p}{:}
            \PY{n}{fig}\PY{o}{.}\PY{n}{savefig}\PY{p}{(}\PY{n}{f\PYZus{}out}\PY{o}{.}\PY{n}{split}\PY{p}{(}\PY{l+s}{'}\PY{l+s}{.}\PY{l+s}{'}\PY{p}{)}\PY{p}{[}\PY{l+m+mi}{0}\PY{p}{]}\PY{o}{+}\PY{l+s}{"}\PY{l+s}{.svg}\PY{l+s}{"}\PY{p}{,}\PY{n}{format}\PY{o}{=}\PY{l+s}{'}\PY{l+s}{svg}\PY{l+s}{'}\PY{p}{)}
\end{Verbatim}

And the Mutual Information Matrix is also saved to disk in text format.

    \begin{Verbatim}[commandchars=\\\{\}]
{\color{incolor}In [{\color{incolor} }]:} \PY{n}{np}\PY{o}{.}\PY{n}{savetxt}\PY{p}{(}\PY{n}{f\PYZus{}out}\PY{p}{,}\PY{n}{M}\PY{p}{)}
        \PY{n}{quit}\PY{p}{(}\PY{p}{)}
\end{Verbatim}


    % Add a bibliography block to the postdoc
    
    
    
    \end{document}
